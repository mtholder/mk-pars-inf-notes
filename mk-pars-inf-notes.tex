\documentclass[11pt]{article}
\usepackage{graphicx}
\DeclareGraphicsRule{.tif}{png}{.png}{`convert #1 `dirname #1`/`basename #1 .tif`.png}
\usepackage{wrapfig}
\textwidth = 6.5 in
\textheight = 9 in
\oddsidemargin = 0.0 in
\evensidemargin = 0.0 in
\topmargin = 0.0 in
\headheight = 0.0 in
\headsep = 0.0 in
\parskip = 0.2in
\parindent = 0.0in
\usepackage{setspace}
\singlespacing

\usepackage{titlesec}
\usepackage{paralist} %compactenum

%\newtheorem{theorem}{Theorem}
%\newtheorem{corollary}[theorem]{Corollary}
%\newtheorem{definition}{Definition}
\usepackage{tipa}
\usepackage{bm}
\usepackage{amsfonts}
\usepackage[mathscr]{eucal}

\titlespacing*{\section}{0cm}{1mm}{-3mm}
\titlespacing*{\subsection}{0cm}{-3mm}{-3mm}
% Use the natbib package for the bibliography
\usepackage[round]{natbib}
\bibliographystyle{apalike} 
\newcommand{\exampleMacro}[1]{\mu_{#1}}
\usepackage{xspace}
\usepackage{url}

\usepackage{algorithm}% http://ctan.org/pkg/algorithms
\usepackage{algpseudocode}% http://ctan.org/pkg/algorithmicx

\usepackage{amsfonts}
\usepackage{hyperref}

\hypersetup{backref,  linkcolor=blue, citecolor=red, colorlinks=true, hyperindex=true}

\newcounter{myFigCounter}
\begin{document}
\tableofcontents
\section{Background}
\citet{Lewis2001} discussed the need to correct likelihood calculations of discrete morphological traits for ascertainment biases that exclude the possibility of a character being constant.
The model that corrects for this ascertainment bias is referred to as the M$k_v$ model.
The correction entails calculating:
$ \Pr(X \mid T, \mbox{each character is variable})$ rather than
$\Pr(X \mid T)$, where $X$ is the data matrix and $T$ is the tree.
\begin{eqnarray*}
\Pr(X \mid T, \mbox{each character is variable}) & = & \prod_{i=1}^M\Pr(x_i \mid T, x_i\mbox{ is variable})
\end{eqnarray*}
where $M$ is the number of characters (columns) in the data matrix.
Fortunately,
$$\Pr(x_i \mid T, x_i\mbox{ is variable})\Pr(x_i\mbox{ is variable}\mid T) = \Pr(x_i\mid T)$$
so:
$$\Pr(x_i \mid T, x_i\mbox{ is variable}) = \frac{\Pr(x_i\mid T)}{\Pr(x_i\mbox{ is variable}\mid T)}$$
$\Pr(x_i\mbox{ is variable}\mid T)$ is typically calculated by augmenting the real data matrix with a set of ``fake'' columns of constant characters:
\begin{eqnarray*}
\Pr(x_i\mbox{ is variable}\mid T) & = & 1 -\Pr(x_i\mbox{ is constant}\mid T) \\
& = & 1 - \sum_{j=1}^K\Pr(C(j)\mid T)
\end{eqnarray*}
where $K$ is the number of character states and $C(j)$ denotes a 
character that is made up of a column of taxa all displaying state $j$.
Because there are only $K$ fake columns for every character, the calculation is only $(1+K)$ times slower than a likelihood calculation that does not correct for ascertainment bias.


As noted by \citet{MatzkeI2018}, if a researcher only collects parsimony informative sites, then a different form of ascertainment bias correction is needed.
One could sum up all of the parsimony uninformative sites:
\begin{eqnarray*}
\Pr(x_i\mbox{ is informative}\mid T) & = & 1 -\Pr(x_i\mbox{ is uninformative}\mid T) \\
& = & 1 - \sum_{j\in \mathcal{U}} \Pr(j\mid T)
\end{eqnarray*}
where $\mathcal{U}$ denotes the set of all columns that are parsimony-uninformative.
However the size of $\mathcal{U}$ grows as a function of the number of tips ($N$).
Indeed it grows quickly for $K>2$.
Here we describe a dynamic programming approach that avoids the need to
	enumerate all uninformative patterns.
This let's us calculate
$\Pr(x_i\mbox{ is uninformative}\mid T)$
more efficiently.

\section*{Algorithm}
For the typical case $N > K$, and a parsimony informative character 
is one that has one repeatedly observed state and a set of singleton states. 
The set of singleton states is the empty set in the case of 
	constant character.
When $N \leq K$, one must consider the possibility that no state is repeated.

Let $\mathcal{A}$ denote the alphabet of states (so $|\mathcal{A}| = K$).
Let, $\mathcal{B}(y)$ denote $\mathcal{A} - y$; this is the set of 
	states other than some state $y$; and
	$\mathcal{B}(y, w)$ denote $\mathcal{A} - w - y$; this is the set of states excluding state $y$ and state $w$.
Let $\mathcal{S}(y)$ be the power set of $\mathcal{B}(y)$.
Similarly $\mathcal{S}(y, w)$ is the power set of $\mathcal{B}(y, w)$.

Below we will suppress the index ($i$ above) that indicates which character we are correcting.
The ascertainment bias correction depends only on the tree and branch length induced by considering leaves that are scored (not missing data).
Thus the calculations do not depend on the specific states of column
$x_i$; so using $x$ is sufficient.

We want to calculate $\Pr(x\mbox{ is uninformative}\mid T) = \Pr(x\in \mathcal{U}\mid T)$
\begin{eqnarray*}
\Pr(x\in \mathcal{U}\mid T) & = & \mathcal{D}(T) + \sum_{r\in \mathcal{A}}\sum_{s\in\mathcal{S}(r)}\Pr(\mbox{r}=r, \mbox{s}=s\mid T)
\end{eqnarray*}
where ``r'' is the random variable representing the repeated state; ``s'' is the random variable representing the singelton state set; and 
$\mathcal{D}(T)$ is the summation (only needed for small trees) over
all patterns with no repeated states:
\begin{eqnarray*}
\mathcal{D}(T) = \sum_{s\in\mathcal{S}(\emptyset)}\Pr(\mbox{r}=\emptyset, \mbox{s}=s\mid T)
\end{eqnarray*}


We can calculate $\Pr(\mbox{r}=r, \mbox{s}=s\mid T)$ via postorder traversal similar to Felsenstein's pruning algorithm.
Let $z$ denote an internal node, and let the lookup table element
$Y[z][r][s][c]$ hold
the probability of observing repeated state $r$ and singleton state set $s$ among the descendants of node $z$ if node $z$ had ancestral
character state $c$.

If $\rho$ denotes the node indicator for the root of the tree, then
\begin{eqnarray*}
\Pr(\mbox{r}=r, \mbox{s}=s\mid T) & = & \sum_{c\in\mathcal{A}} \pi_c Y[\rho][r][s][c]\\
\end{eqnarray*}
where $\pi_c$ is the root prior probability of state $c$ (which is
usually the equilibrium state frequency of $c$).


Let $l(z)$ and $r(z)$ denote the the left and right children of node($z$).
Let $\nu_{l(z)}$ and $\nu_{r(z)}$ denote the  branch lengths leading to the left and right children of node($z$).

\subsection{Parent of a cherry}
If $z$ is the parent of two tips then initialize $Y[z][][][]$ using the following rules

For every $c\in \mathcal{A}$:
\begin{eqnarray*}
Y[z][c][\emptyset][c] & = & \Pr(c\rightarrow c\mid \nu_{l(z)})\Pr(c\rightarrow c\mid \nu_{r(z)})
\end{eqnarray*}
where  $\Pr(i\rightarrow j\mid \nu_{k})$ denotes the probability of an 
	seeing the descendant state $j$ if the ancestral state is $i$ and 
	the edge length is $\nu_{k}$.


For every $c\in \mathcal{A}$ and every single state $d \in \mathcal{B}(c)$:
\begin{eqnarray*}
Y[z][\emptyset][\{c,d\}][c] & = & \Pr(c\rightarrow c\mid \nu_{l(z)})\Pr(c\rightarrow d\mid \nu_{r(z)}) + \Pr(c\rightarrow d\mid \nu_{l(z)})\Pr(c\rightarrow c\mid \nu_{r(z)}) \\
Y[z][d][\emptyset][c] & = & \Pr(c\rightarrow d\mid \nu_{l(z)})\Pr(c\rightarrow d\mid \nu_{r(z)})
\end{eqnarray*}

For every $c\in \mathcal{A}$ and every single state $d \in \mathcal{B}(c)$ and every single state $f\in\mathcal{B}(c,d)$:
\begin{eqnarray*}
Y[z][\emptyset][\{d,f\}][c] & = & \Pr(c\rightarrow d\mid \nu_{l(z)})\Pr(c\rightarrow f\mid \nu_{r(z)}) + \Pr(c\rightarrow f\mid \nu_{l(z)})\Pr(c\rightarrow d\mid \nu_{r(z)})
\end{eqnarray*}

All other elements of $Y[z][][][]$ are set to 0.

\subsection{Parent of two internals}
If $z$ is the parent of two other internal nodes do the following.

For every $r\in \mathcal{A}$, $c\in \mathcal{A}$ and $s\in\mathcal{S}(c)$:
\begin{eqnarray*}
Y[z][r][s][c] & = & \sum_{d\in \mathcal{A}}Pr(c\rightarrow d\mid \nu_{l(z)})\sum_{f\in \mathcal{A}}Pr(c\rightarrow f\mid \nu_{r(z)}) V(z, r, s, d, f) \\
V(z, r, s, d, f) & := &\sum_{t\in \mathcal{S}(r)}\Bigg(Y[l(z)][r][t][d] Y[r(z)][r][s - t][f]
\\ 
& & \ldots + Y[l(z)][r][t][d] Y[r(z)][\emptyset][s - t][f] \\
& & \ldots + Y[l(z)][r][t][d] Y[r(z)][\emptyset][r + s - t][f]\\
& & \ldots + Y[l(z)][\emptyset][t][d] Y[r(z)][r][s - t][f] \\
& & \ldots + Y[l(z)][\emptyset][r + t][d] Y[r(z)][r][s - t][f]\\
& & \ldots + Y[l(z)][\emptyset][r + t][d] Y[r(z)][\emptyset][r + s - t][f] \Bigg)
\end{eqnarray*}

\bibliography{refs}
\end{document}
